\documentclass{article}
\usepackage{amsmath,amssymb}
\usepackage{bm}
\usepackage[margin=1in]{geometry}
\usepackage{graphicx}
\usepackage{hyperref}
\usepackage[latin1]{inputenc}
\usepackage{units}

% paragraph layout
\setlength{\parindent}{0pt}
\setlength{\parskip}{2ex plus 0.8ex minus 0.5ex}

% derivatives
\newcommand{\dd}[2]{\frac{d #1}{d #2}}

\newcommand{\for}[0]{\quad \text{ for } \quad}
\newcommand{\xli}[0]{x_{i-1/2}}
\newcommand{\xri}[0]{x_{i+1/2}}
\newcommand{\xlj}[0]{x_{j-1/2}}
\newcommand{\xrj}[0]{x_{j+1/2}}
\newcommand{\vli}[0]{v_{i-1/2}}
\newcommand{\vri}[0]{v_{i+1/2}}

\numberwithin{equation}{section}

% XXX: indexing conventions?

%%%%%%%%%%%%%%%%%%%%%%%%%%%%%%%%%%%%%%%%%%%%%%%%%%%%%%%%%%%%%%%%%%%%%%
% title page

\begin{document}

\title{Weighted essentially non-oscillatory schemes}
\author{Matthew Emmett}
\maketitle

%%%%%%%%%%%%%%%%%%%%%%%%%%%%%%%%%%%%%%%%%%%%%%%%%%%%%%%%%%%%%%%%%%%%%%
% introduction

\section{Introduction}

We follow C.W. Shu in ``Essentially Non-oscillatory and Weighted
Essentially Non-oscillatory Schemes for Hyperbolic Conservation Laws''
(NASA/CR-97-206253, ICASE report no. 97-65).

We consider a one-dimensional hyperbolic conservation law of the form
\begin{gather}
  \label{eq:pdecons}
  q_t + \bigl( f(q) \bigr)_x = 0.
\end{gather}
For finite-volume schemes we do not solve \eqref{eq:pdecons} directly,
but its integrated version instead.  Integrating \eqref{eq:pdecons}
over the interval $[a,b]$ we obtain
\begin{equation*}
  \dd{}{t} \overline{q}(t)
    + \frac{1}{b - a} \Bigl( f\bigl(q(b,t)\bigr)
                                - f\bigl(q(a,t)\bigr) \Bigr) = 0
\end{equation*}
where
\begin{equation*}
  \overline{q}(t)
    \equiv \frac{1}{b - a} \int_{a}^{b} q(\xi,t) \;d\xi
\end{equation*}
is the average value of $q$ over $[a,b]$.  This leads us to one of
the central problems in implementing a numerical scheme to solve
\eqref{eq:pdecons}: obtaining the values of $q$ at the boundaries $a$
and $b$ based on the average $\overline{q}$ of $q$.  This is the
\emph{reconsturction} problem.

It is our intention here to study the reconstruction problem.

\newpage
\section{Grid}

We consider a grid over the interval $[a,b]$ with $N$ cells.  We
denote the $N+1$ cell boundaries by
\begin{equation}
  x_{i-1/2} \for i = 1,\ldots,N+1
\end{equation}
so that
\begin{equation}
  a = x_{1/2} < x_{3/2} < \cdots < x_{N-1/2} < x_{N+1/2} = b.
\end{equation}
Subsequently, we denote the $N$ cells by
\begin{equation}
  C_i = [\xli, \xri] \for i=1,\ldots,N;
\end{equation}
the $N$ cell centres by
\begin{equation}
  x_i = \frac{\xli + \xri}{2} \for i=1,\ldots,N;
\end{equation}
the $N$ cell sizes by
\begin{equation}
  \Delta x_i = \xri - \xli \for i=1,\ldots,N;
\end{equation}
and the maximum cell size by
\begin{equation}
  \Delta x = \max_{i=1,\ldots,N} \Delta x_i.
\end{equation}

We denote the contiguous stencil around the cell $C_i$, containing $k$
cells shifted to the left by $r$ cells, by
\begin{equation}
  S_i^{r,k} = C_{i-r} \cup \cdots \cup C_{i-r+k-1}.
\end{equation}
Note that $S_i^{r,k}$ spans $k$ cells and contains $k+1$ cell
boundaries.

\newpage
\section{One dimensional reconstruction for smooth functions}
\label{sec:reconstruction}

Given the cell averages $\overline{v}_j$ of a function $v$ where
\begin{equation}
  \overline{v}_j = \frac{1}{\Delta x_j} \int_{\xlj}^{\xrj} v(\xi) \;d\xi
\end{equation}
we wish to find approximations $\vri$ to the function $v$ at the cell
boundaries $\xri$, based on $k$ cell averages, that are $k$-order
accruate.  The remainder of this section will be devoted to finding
these approximations and showing that they are $k$-order accuarate.
As it turns out, we will show that there are constants $c^r_{ij}$
(hereafter called \emph{reconstruction coefficients}) such that the
reconstructed values $\vri$ at the cell boundaries $\xri$ are given by
\begin{equation*}
  \vri = \sum_{j=0}^{k-1} c^r_{ij}\; \overline{v}_{i-r+j}.
\end{equation*}
That is, given a stencil $S_i^{r,k}$ that spans the $k$ cells
$C_{i-r},\ldots,C_{i-r+k-1}$, the reconstructed value $\vri$ of the
original function $v$ at the cell boundary $\xri$ can be obtained
using the cell averages $\overline{v}_{j}$ over the cells $C_j$ in the
stencil $S_i^{r,k}$.  In general, the reconstruction coefficients
depend on the order $k$, left shift $r$, and cell $i$.

We can also reconstruct the values $\vli$ using the reconstruction
coefficients $\tilde{c}^r_{ij}$ so that
\begin{equation}
  \label{eq:vli1}
  \vli = \sum_{j=0}^{k-1} \tilde{c}^r_{ij}\; \overline{v}_{i-r+j}
\end{equation}
As it turns out, the reconstruction coefficients $c^r_{ij}$ and
$\tilde{c}^r_{ij}$ are related.  This can be seen by considering the
cell $C_{i-1}$ and the stencil $S_{i-1}^{r-1,k}$.  Then
\begin{equation}
  \label{eq:vli2}
  \vli = v_{(i-1)+1/2}
    = \sum_{j=0}^{k-1} c^{r-1}_{(i-1)\,j}\; \overline{v}_{(i-1)-(r-1)+j}
    = \sum_{j=0}^{k-1} c^{r-1}_{(i-1)\,j}\; \overline{v}_{i-r+j}
\end{equation}
Comparing \eqref{eq:vli1} and \eqref{eq:vli2}, we see that
\begin{equation*}
  \tilde{c}^r_{ij} = c^{r-1}_{(i-1)\, j}.
\end{equation*}

In order to obtain the reconstruction coefficients $c^r_{ij}$ and
prove accuracy, we first consider the following approximation problem:
Given the cell averages $\overline{v}_j$ of a function $v$, find the
polynomials $p^r_i$ of degree at most $k-1$ such that each $p^r_i$ is
a $k$-order accurate approximation to $v$ inside $C_i$.  That is,
given the cell averages $\overline{v}_j$, find polynomials $p^r_i$
such that
\begin{equation*}
  p^r_i(x) = v(x) + O(\Delta x^k) \for x \in C_i.
\end{equation*}

In order to find these polynomials, we consider the function
\begin{equation}
  \label{eq:V}
  V(x) = \int_a^x v(\xi) \;d\xi.
\end{equation}
Using the cell averages $\overline{v}_j$ we can compute $V$ at the
cell boundaries $\xri$ through
\begin{align}
  V(\xri) &= \int_a^{\xri} v(\xi) \;d\xi \nonumber \\
          &= \sum_{j=1}^i \int_{\xlj}^{\xrj} v(\xi) \;d\xi \nonumber \\
          &= \sum_{j=1}^i \overline{v}_j \Delta x_j.
          \label{eq:Vsum}
\end{align}

Focusing on a particular cell $C_i$ and stencil $S_i^{r,k}$, the
unique polynomial $P^r_i$ of order $k$ which interpolates $V$ at the
$k+1$ points
\begin{equation*}
  x_{i-r-1/2}, \ldots, x_{i-r+k-1/2}
\end{equation*}
is given by
\begin{equation}
  \label{eq:P}
  P^r_i(x) = \sum_{l=0}^{k} \Biggl(
    V(x_{i-r+l-1/2}) \prod_{m=0, m \neq l}^{k}
      \frac{(x - x_{i-r+m-1/2})}{(x_{i-r+l-1/2} - x_{i-r+m-1/2})}
    \Biggr).
\end{equation}
(This is the interpolating polynomial of $V$ in Lagrange form.)  We
denote the derivative of $P^r_i$ by $p^r_i$, so that
\begin{equation*}
  p^r_i(x) = \dd{}{x} P^r_i(x).
\end{equation*}
It can be shown (see Appendix~\ref{app:lagrange}) that
\begin{equation*}
  P^r_i(x) = V(x) + O(\Delta x^{k+1}) \for x \in S_i^{r,k}.
\end{equation*}
Therefore
\begin{equation*}
  p^r_i(x) = v(x) + O(\Delta x^k) \for x \in S_i^{r,k}
\end{equation*}
and $p^r_i(x)$ is of order $k-1$.

Furthermore, the cell averages of $p^r_i$ over the cells $C_j$ that
comprise the stencil $S_i^{r,k}$ satisfy
\begin{align*}
  \frac{1}{\Delta x_j} \int_{\xlj}^{\xrj} p^r_i(\xi) \;d\xi
    &= \frac{1}{\Delta x_j} \int_{\xlj}^{\xrj} P'_i(\xi) \;d\xi \\
    &= \frac{1}{\Delta x_j} \biggl( P^r_i(\xrj) - P^r_i(\xlj) \biggr) \\
    &= \frac{1}{\Delta x_j} \biggl( V(\xrj) - V(\xlj) \biggr) \\
    &= \frac{1}{\Delta x_j} \biggl( \int_a^{\xrj} v(\xi) \;d\xi
      - \int_a^{\xlj} v(\xi) \;d\xi \biggr) \\
    &= \frac{1}{\Delta x_j} \biggl(
      \int_{\xlj}^{\xrj} v(\xi) \;d\xi \biggr) \\
    &= \overline{v}_j \for j=i-r,\ldots,i-r+k-1.
\end{align*}
That is, the cell averages of the approximating polynomials $p^r_i$
match the cell averages the original function $v$ in each of the cells
$C_j$ which comprise the stencil $S_i^{r,k}$.

So far we have constructed polynomials $p^r_i$ that approximate the
original function $v$ on the stencils $S_{i}^{r,k}$ to $k$-order using
only the cell averages $\overline{v}_j$ for $j=i-r,\ldots,i-r+k-1$.

Now we consider the practical problem of finding the constants
$c^r_{ij}$.  Subtracting $V(x_{i-r-1/2})$ from $P^r_i(x)$ and using
\begin{equation*}
  \sum_{l=0}^{k} \prod_{m=0, m \neq l}^{k}
    \frac{(x - x_{i-r+m-1/2})}{(x_{i-r+l-1/2} - x_{i-r+m-1/2})} = 1
  \quad \text{ and } \quad
  V(x_{i-r+l-1/2}) - V(x_{i-r-1/2}) \equiv 0 \text{ for } l = 0
\end{equation*}
we obtain
\begin{equation*}
  P^r_i(x) - V(x_{i-r-1/2}) = \sum_{l=1}^{k} \Biggl(
    \bigl( V(x_{i-r+l-1/2}) - V(x_{i-r-1/2}) \bigr)
    \prod_{m=0, m \neq l}^{k}
      \frac{(x - x_{i-r+m-1/2})}{(x_{i-r+l-1/2} - x_{i-r+m-1/2})}
      \Biggr).
\end{equation*}
Taking the derivative of the above, we obtain
\begin{equation*}
  \dd{}{x}P^r_i(x) = \dd{}{x} \Biggl[ \sum_{l=1}^{k} \Biggl(
    \bigl( V(x_{i-r+l-1/2}) - V(x_{i-r-1/2}) \bigr) \prod_{m=0, m \neq l}^{k}
      \frac{(x - x_{i-r+m-1/2})}{(x_{i-r+l-1/2} - x_{i-r+m-1/2})}
      \Biggr) \Biggr]
\end{equation*}
and hence
\begin{equation}
  p^r_i(x) = \sum_{l=1}^{k} \Biggl(
  \bigl( V(x_{i-r+l-1/2}) - V(x_{i-r-1/2}) \bigr) \
    \frac{\sum_{m=0, m \neq l}^{k} \prod_{n=0, n \neq l,m}^{k}
      (x - x_{i-r+n-1/2})}{\prod_{m=0, m \neq l}^{k}
      (x_{i-r+l-1/2} - x_{i-r+m-1/2})} \Biggr). \label{eq:pV}
\end{equation}
Evaluing $p^r_i$ at the cell boundary $\xri$ and employing
\eqref{eq:Vsum}, we obtain
\begin{equation}
  p^r_i(\xri) = \sum_{l=1}^{k} \Biggl(
    \biggl( \sum_{j=0}^{l-1} \overline{v}_{i-r+j} \Delta x_{i-r+j} \biggr)
    \frac{\sum_{m=0, m \neq l}^{k} \prod_{n=0, n \neq l,m}^{k}
      (\xri - x_{i-r+n-1/2})}{\prod_{m=0, m \neq l}^{k}
      (x_{i-r+l-1/2} - x_{i-r+m-1/2})} \Biggr).
\end{equation}
Rearranging, we obtain
\begin{equation*}
  \vri = p^r_i(\xri) = \sum_{j=0}^{k-1} \sum_{l=j+1}^k
    \frac{\sum_{m=0, m \neq l}^{k} \prod_{n=0, n \neq l,m}^{k}
      (\xri - x_{i-r+n-1/2})}{\prod_{m=0, m \neq l}^{k}
      (x_{i-r+l-1/2} - x_{i-r+m-1/2})} \ \Delta x_{i-r+j} \overline{v}_{i-r+j}.
\end{equation*}
Therefore, the reconstruction coefficients $c^r_{ij}$ are given by
\begin{equation}
  c^r_{ij} = \sum_{l=j+1}^k \frac{\sum_{m=0, m \neq l}^{k}
    \prod_{n=0, n \neq l,m}^{k} (\xri - x_{i-r+n-1/2})}{
    \prod_{m=0, m \neq l}^{k} (x_{i-r+l-1/2} - x_{i-r+m-1/2})}
    \ \Delta x_{i-r+j}
\end{equation}
and depend on $r$ and $k$.

% XXX: note re: c(\xi)...

\subsection{Further derivatives}

In order to approximate the spatial derivative $v'(x)$ we consider the
first derivative of $p^r_i(x)$.  We obtain
\begin{align*}
  \dd{}{x} p^r_i(x) &= \dd{}{x} \Biggl[ \sum_{l=1}^{k} \Biggl(
  \bigl( V(x_{i-r+l-1/2}) - V(x_{i-r-1/2}) \bigr) \
    \frac{\sum_{m=0, m \neq l}^{k} \prod_{n=0, n \neq l,m}^{k}
      (x - x_{i-r+n-1/2})}{\prod_{m=0, m \neq l}^{k}
      (x_{i-r+l-1/2} - x_{i-r+m-1/2})} \Biggr) \Biggr] \\
    &= \sum_{l=1}^{k} \Biggl(
    \bigl( V(x_{i-r+l-1/2}) - V(x_{i-r-1/2}) \bigr) \
    \frac{\sum_{m=0, m \neq l}^{k} \sum_{n=0, n \neq l,m}^{k} \prod_{p=0, p \neq l, m, n}
      (x - x_{i-r+p-1/2})}{\prod_{m=0, m \neq l}^{k}
      (x_{i-r+l-1/2} - x_{i-r+m-1/2})} \Biggr).
\end{align*}
Evaluating the above at $x = \xi$ and employing \eqref{eq:Vsum}, we
obtain
\begin{equation*}
  v'(\xi) \approx \dd{}{x} p^r_i(\xi) = \sum_{l=1}^{k} \Biggl(
    \biggl( \sum_{j=0}^{l-1} \overline{v}_{i-r+j} \Delta x_{i-r+j} \biggr)
    \frac{\sum_{m=0, m \neq l}^{k} \sum_{n=0, n \neq l,m}^{k} \prod_{p=0, p \neq l, m, n}
      (\xi - x_{i-r+p-1/2})}{\prod_{m=0, m \neq l}^{k}
      (x_{i-r+l-1/2} - x_{i-r+m-1/2})} \Biggr).
\end{equation*}
Rearranging, we obtain
\begin{equation*}
  v'(\xi) \approx \dd{}{x} p^r_i(\xi) = \sum_{j=0}^{k-1} \sum_{l=j+1}^k
  \frac{\sum_{m=0, m \neq l}^{k} \sum_{n=0, n \neq l,m}^{k} \prod_{p=0, p \neq l, m, n}
    (\xi - x_{i-r+p-1/2})}{\prod_{m=0, m \neq l}^{k}
    (x_{i-r+l-1/2} - x_{i-r+m-1/2})}
  \ \Delta x_{i-r+j} \overline{v}_{i-r+j}.
\end{equation*}
Therefore, the reconstruction coefficients for the first derivative
are
\begin{equation*}
  c_j = \sum_{l=j+1}^k
  \frac{\sum_{m=0, m \neq l}^{k} \sum_{n=0, n \neq l,m}^{k} \prod_{p=0, p \neq l, m, n}
    (\xi - x_{i-r+p-1/2})}{\prod_{m=0, m \neq l}^{k}
    (x_{i-r+l-1/2} - x_{i-r+m-1/2})}
  \ \Delta x_{i-r+j}
\end{equation*}

\subsection{Summary}

In summary, given a stencil $S_i^{r,k}$ and the cell averages
$\overline{v}_j$ of a function $v$, the approximation $\vri$ to the
function $v$ at the cell boundary $\xri$ is given by
\begin{equation}
  \label{eq:sumvri}
  \vri = \sum_{j=0}^{k-1} c^r_{ij}\; \overline{v}_{i-r+j}
\end{equation}
where
\begin{equation}
  c^r_{ij} = \sum_{l=j+1}^k \frac{\sum_{m=0, m \neq l}^{k}
    \prod_{n=0, n \neq l,m}^{k} (\xri - x_{i-r+n-1/2})}{
    \prod_{m=0, m \neq l}^{k} (x_{i-r+l-1/2} - x_{i-r+m-1/2})}
    \ \Delta x_{i-r+j}.
\end{equation}
Furthermore, the approximation $\vri$ is accurate to order $k$ so that
\begin{equation*}
  \vri = v(\xri) + O(\Delta x^k)
\end{equation*}
where
\begin{equation*}
  \Delta x = \max_{j=i-r,\ldots,i-r+k-1} \Delta x_j.
\end{equation*}
Finally, the approximation $\vli$ to the function $v$ at the cell
boundary $\xli$ is given by
\begin{equation*}
  \vli = \sum_{j=0}^{k-1} c^{r-1}_{(i-1)\,j}\; \overline{v}_{i-r+j}.
\end{equation*}
The permissable values of left shift parameter $r$ in
\eqref{eq:sumvri} are $-1,\ldots,k-1$ so that the results of
Appendix~\ref{app:lagrange} hold.

\newpage
\section{One dimensional reconstruction for piece-wise smooth functions}

The solutions of hyperbolic conservation laws may contain
discontinuities, and therefore we are interested in reconstructing
piecewise smooth functions.  A piecewise smooth function $v$ is smooth
except at finitely many isolated points.  At these points, $v$ and its
derivatives (at least up to the order of the scheme) are assumed to
have finite left and right limits.

For such piecewise smooth functions, the order of accuracy herein
referred to is formal- that is, it is defined as the accuracy
determined by the local error in the smooth regions of the function.

The basic idea of WENO is to use a convex combination of several
stencils to form the reconstruction of $v$ at the cell boundaries,
and, if a stencil contains a discontinuity, its weight should be close
to zero.  In smooth regions, using several stencils will also serve to
increase the order of accuracy.

Consider the $k$ stencils
\begin{equation*}
  S_i^{r,k} \for r=0,\ldots,k-1
\end{equation*}
that can be used to reconstruct the value of $v$ at the cell
boundaries $\xli$ and $\xri$.  These stencils span $2k-1$ cells.  We
denote the $k$ different reconstructions by
\begin{equation}
  \label{eq:vr}
  \vri^r = \sum_{j=0}^{k-1}\; c_{ij}^r \bar{v}_{i-r+j} \for r=0,\dots,k-1
\end{equation}
where we have added the superscript $r$ to $\vri$ to make the
dependance upon the left shift $r$ explicit.

A WENO reconstruction takes a convex combination of all $\vri^r$
defined in \eqref{eq:vr} as a new approximation to $\vri$ according to
\begin{equation}
  \label{eq:vweno}
  \vri = \sum_{r=0}^{k-1} \omega_i^r \vri^r
\end{equation}
where we require
\begin{equation}
  \omega_i^r \geq 0 \quad \text{ and } \quad \sum_{r=0}^{k-1} \omega_i^r = 1.
\end{equation}

In smooth regions where all $k$ stencils that can be used to
reconstruct $\vri$ in \eqref{eq:vr} do not contain discontinuities, we
could reconstruct $\vri$ to order $2k-1$ using the stencil
$S_i^{k-1,2k-1}$ to obtain
\begin{equation}
  \label{eq:vopt}
  \vri = \sum_{j=0}^{2k-2} c_{ij}^* \bar{v}_{i-(k-1)+j}.
\end{equation}
Combining \eqref{eq:vr}, \eqref{eq:vweno}, and \eqref{eq:vopt}, we
obtain
\begin{equation}
  \label{eq:vopt2}
  \sum_{j=0}^{2k-2} c_{ij}^* \bar{v}_{i-(k-1)+j}
    = \sum_{r=0}^{k-1} \omega_i^r \left( \sum_{l=0}^{k-1} c_{il}^r \bar{v}_{i-r+l} \right).
\end{equation}
Rearranging, we obtain
\begin{equation*}
  \sum_{j=0}^{2k-2} c_{ij}^* \bar{v}_{i-(k-1)+j}
    = \sum_{j=0}^{2k-2} \left(
      \sum_{l=\max(0,j-k+1)}^{\min(k-1,j)} \omega_i^{k-(j+1)+l} c_{il}^{k-(j+1)+l}
    \right) \bar{v}_{i-(k-1)+j}
\end{equation*}
and hence we obtain systems of the $2k-1$ equations
\begin{gather}
  \label{eq:omegasys}
  \sum_{l=\max(0,j-k+1)}^{\min(k-1,j)} \omega_i^{k-(j+1)+l} c_{il}^{k-(j+1)+l} = c_{ij}^*
    \for j = 0,\ldots,2k-2
\end{gather}
at each $i$ for the weights $\omega_i^r$.  For unstructured grids the
systems \eqref{eq:omegasys} are over-determined, and thereefore we
must use some kind of optimisation algorithm in order to find the
weights $\omega_i^r$.  For structured grids the systems
\eqref{eq:omegasys} are no longer over-determined, and the weights
$\omega_i^r$ can be found explicity (and are independent of $i$).

The weights $\omega_i^r$ defined by \eqref{eq:vopt2} and determined by
\eqref{eq:omegasys} are called \emph{optimal weights} since they can
be used to reconstruct $\vri$ to order $2k-1$ in regions where $v$ is
smooth.  We will henceforth denote the optimal weights by
$\varpi_i^r$.

A similar procedure can be used to determine the optimal weights for
reconstructing $\vli$ on the stencil $S_i^{r,k}$.

We now consider the practical problem of choosing the weights
$\omega_i^r$.  If we choose the weights $\omega_i^r$ sufficiently
close to the optimal weights $\varpi_i^r$ in regions where $v$ is
smooth then we can achieve $2k-1$ order accuracy.  In order to
determine how close to the optimal weights $\varpi_i^r$ the weights
$\omega_i^r$ must be choosen we consider the reconstruction
\begin{equation}
  \label{eq:omegacloseeqn}
  \vri = \sum_{r=0}^{k-1} \omega_i^r \vri^r
    = \sum_{r=0}^{k-1} \varpi_i^r \vri^r
    + \sum_{r=0}^{k-1} \bigl( \omega_i^r - \varpi_i^r \bigr) \vri^r.
\end{equation}
If we choose
\begin{equation}
  \label{eq:omegaclose}
  \omega_i^r = \varpi_i^r + O(\Delta x^{k-1})
\end{equation}
then each term in the last summation of \eqref{eq:omegaclose} becomes
$O(\Delta x^{2k-1})$ hence $2k-1$ order accuracy is preserved by the
reconstruction.

If we define
\begin{equation}
  \label{eq:omega}
  \omega_i^r = \frac{\alpha_i^r}{\alpha_i^0 + \cdots + \alpha_i^{k-1}}
\end{equation}
where
\begin{equation}
  \label{eq:alpha}
  \alpha_i^r = \frac{\varpi_i^r}{(\epsilon + \sigma_i^r)^p} \for r = 0, \ldots, k-1;
\end{equation}
and $\epsilon$ is a positive real number used to avoid dividing by
zero (usually $\epsilon = 10^{-6}$), $p$ is some power, and
$\sigma_i^r$ is a measure of the smoothness of the function $v$ in the
stencil $S^{r,k}_i$; with the smoothnesses $\sigma_i^r$ chosen
appropriately, then \eqref{eq:omegaclose} is satisfied.

% We use the smoothness measurement presented by Jiang and Shu:
% \begin{equation}
%   \label{eq:sigma}
%   \sigma_i^r = \sum_{l=1}^{k-1} \int_{\xlj}^{\xrj} (\Delta x_j)^{2l-1} \left( \frac{d^l}{dx^l} p_i^r  \right)^2 \;dx
% \end{equation}

% Note that
% \begin{align*}
%   \int_{\xlj}^{\xrj} \left( \frac{d^l}{dx^l} p_i^r  \right)^2 \;dx =
% \end{align*}



%%
%% appendix
%%

\newpage
\appendix
\section{Error of Lagrange interpolating polynomials}
\label{app:lagrange}

Let $f(x) \in C^{n}([a,b])$, and $p(x)$ be the interpolating
polynomial of degree $n-1$ such that
\begin{equation}
  p(x_i) = f(x_i) \for i=1,\ldots,n
\end{equation}
where
\begin{equation}
  a = x_1 < x_2 < \cdots < x_{n-1} < x_n = b.
\end{equation}
Then
\begin{equation}
  p(x) = f(x) + O(\Delta x^n) \for x \in [a,b]
\end{equation}
where
\begin{equation}
  \Delta x = \max_{i=2,\ldots,n} x_i - x_{i-1}.
\end{equation}

\textbf{Proof.}  Let $x \in [a,b]$.  If $x = x_i$ for some
$i=1,\ldots,n$ then $f(x) - p(x) = 0$ since $p(x)$ is the
interpolating polynomial.  Otherwise, let
\begin{equation}
  \Phi(x) = \frac{f(x) - p(x)}{\prod_{i=1}^n (x - x_i)}
\end{equation}
and
\begin{equation}
  g(x,\xi) = f(\xi) - p(\xi) - \Phi(x) \prod_{i=1}^n (\xi - x_i).
\end{equation}
Then $g(x,\xi)$ is $n$ times differentiable with respect to $\xi$,
$g(x,x_i) = 0$ for $i=1,\ldots,n$, and $g(x,x) = 0$.  Applying Rolle's
theorem successively across all interpolation points $x_i$ and $x$ we
obtain
\begin{equation}
  \label{eq:xistar}
  \frac{\partial^n}{\partial \xi^n} g(x,\xi)\biggr|_{\xi=\xi^*} = 0
\end{equation}
for some $\xi^* \in (a,b)$.  Futhermore
\begin{equation}
  \label{eq:partials}
  \frac{\partial^n}{\partial \xi^n} g(x,\xi) = f^n(\xi) - n!\;\Phi(x)
\end{equation}
so that, combininig \eqref{eq:xistar} and \eqref{eq:partials}, we obtain
\begin{equation}
  \Phi(x) = \frac{f^n(\xi^*)}{n!}
\end{equation}
and therefore
\begin{equation}
  f(x) - p(x) = \frac{f^n(\xi^*)}{n!} \prod_{i=1}^n (x - x_i).
\end{equation}
Finally, we conclude that
\begin{equation}
  p(x) = f(x) + O(\Delta x^n).
\end{equation}
That is, if $p(x)$ interpolates $f(x)$ at $n$ points, then it is
accurate to $O(\Delta x^n)$ where $\Delta x$ is the maximum space
between the interpolating points.

\end{document}
